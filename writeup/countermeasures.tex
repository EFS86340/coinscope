\section{Countermeasures}
Bitcoin is widely associated as providing ``anonymity'' and privacy, although in the case of private transactions, it has been known that there is very little guarantee, since transactions can easily be clustered and associated. Our techniques show that this is also the case for the connectivity of the network itself.

It may be desirable for the Bitcoin to network to prevent the sort of analysis we have undertaken and to obscure its structure. Our techniques exploited .... \anote{some generalization about how our techniques work}

\anote{What can we actually do with this information? Are users exposed to attacks?}o

\paragraph{Randomization.}
Don't always send transactions immediately? Increase the number of transactions necessary?
