\subsection{Network Mapping with AddrProbe}

While the TxProbe techniques work by exploiting properties of the transaction propagating subsystem, our second set of techniques takes advantage of the address propagating subsystem.

\paragraph{Dumping a Node's \ds{addrMan}.}
We can read the contents of a node's \ds{addrMan} by sending it repeated \msg{GETADDR} messages.
\anote{How long does it take? How do we know when to stop? What precision do we get?}

\subsubsection{Inferring Connections from \ds{addrMan} Dumps}

After dumping the addrMan of every reachable node in the network, we process the data to infer.
We process the data by first building table of frequencies of timestamps. $(ts,y) \in reports[x]$ means we received an $\msg{ADDR}$ message from node $y$ containing the address of $x$, with the timestamp $ts$. The $x.reports[ts]$ contains the frequency of reports about $x$ with the timestamp $ts$.

If a node reports another with a timestamp less than $<2$ hrs, and that timestamp has low frequency (e.g., one or two) then we infer it is an \emph{inbound} connection.
\anote{much more to fill in here}

\begin{table}[h!]
\centering
\caption{Connetion Inference rules for AddrProbe \anote{this is good in code, but hard to read.}}
\vspace{3pt}
\begin{tabular}{|l||*{5}{c|}}
\hline 
\backslashbox{$A$}{$B$} & $U<2h$ & $>2h$ & $ \neg U < 2h$ & $\emptyset$ \\
\hline & \\[-1.05em]\hline  $U<2h$ & $A \leftrightarrows B$ & $A \rightarrow B$ & $A \rightarrow B$ & $\not~$ \\
\hline $>2h$ & $A \leftarrow B$ & $\not$ & $A \not \rightarrow B$ & $\not~$ \\
\hline $\neg U < 2h$ & $A \leftarrow B$  & $A \not \leftarrow B$  & $?$ & $\not~$ \\
\hline $\emptyset$ & $\not$ & $\not$ & $\not$ & $\not~$
\\ \hline

\end{tabular}
\end{table}

\subsubsection{The technique of CryptoLux}
Our AddrProbe techniques are related to (but different than) techniques described in ~\cite{cryptolux}, which also exploit behavior of the address propagation subsystem. We briefly review their technique.

Their technique involves sending $N$ bogus ``marker'' addresses to a target node. The marker addresses contain a timestamp of nearly ten minutes in the past, which prevents them from being propagated more than one or two hops. By looking for these marker addresses on the network, we can estimate two things. First, by listening for the marker addresses relayed back to the injection node, we can estimate the degree of the node, since each address is relayed to one of the peers chosen at random. Next by reading the contents of other nodes addrMan and counting the number of marker addresses that occur, we can estimate whether a node is or is not a peer of the taret node. If the target node has degree $D$, then we expect on average $N/D$ marker addresses to appear in the peer nodes addrMan. On the other hand, if the peer is two hops away, we expect it to have $(N/D)*(N/D_0)$ (where $D_0$ is the degree of the peer) of the markers.
