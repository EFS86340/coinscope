\section{Evaluation of our techniques}

\subsubsection{Causes of Errors}
What can go wrong?

\paragraph{Robustness to Incomplete Connections}
\anote{What do we do when the topology changes during our experiment? What are the possible cases? Which cases affect more than just a single nodes connectivity - is it possible that a disconnect at the wrong time causes a key transaction to escape, resulting in a cascade of false positives? We really need to work through this.}


\paragraph{Blocks Interruptions.} Blocks can arrive during the middle of a txprobe trial. Often, the block will contain one or more of the transactions in our test. This can include \anote{the Flood tx? the Parent tx?} \anote{Effect on cost of the transaction?}

\paragraph{Late Arrivals.} If a peer joins the network \emph{after} the flood transaction has been transmitted, then the peer will a) not be inv-locked, and b) not have the flood. Therefore, this peer may obtain one of the parent transactions, and therefore act as a conduit. This can cause false positives. \anote{Poor writing, fix.}

\subsubsection{Costs}
How much does it cost to perform a txprobe? It depends on which transactions get included in blocks.

We evaluated our attacks and network mapping techniques using a both simulations as well as cross-validation and comparison with ground truth on the live Bitcoin network.

\subsection{Simulations}


\subsection{Ground Truth}
We collected groundtruth from several nodes running the standard Bitcoin client. Some (5) of these nodes were run by us, others were nodes operated by Bitcoin users or developers we asked. 

\subsection{Cross Validation}
We cross validated the inferred edges (and non-edges) from both of our methods.
